								Analysis of using capital asset pricing model for assessing companies return
																Review

	Introduction

The article begin by introducing the impact of risked investment on the russian market. 
The risk constitues an uncertainty which prevent the researchers from probability calculations, necessary to calculate the Capital Pricing Model (CAPM)

	Methods
%Assess = Evaluer
%asset = actif
%Difference between Capital and equity ? 
%Shareholders' equity = Capitaux des ationnaires ou capitaux propres ? 
ROE : Return on equity 
	->Net income / Shareholders' equity
	->Return on net assets
	%Parce que les capitaux propres sont egaux aux actifs d'une société - ses dettes, le ROE est considéré comme la rentabilité de ses 		capitaux propres

According to this article, the CAPM is the most common method to calculate the ROE.
However, because of their instability, emergent markets think that the CAPM is innapropriate and inefficient.
The searchers estimate than the cause is a wrong beta coefficient assessment, ignoring caracteristics of the local market.
The accounting beta is estimated in regression of company earnings to company portfolio earning.
%(Le bétâ comptable est estimé en régression des bénéfices de l'entreprise par rapport aux revenus du portfeuille de l'entreprise)

The study used the MTS company as an example and the Bloomberg data to complete their results tab

Conclusion

I thought that this article would give precisions on the factors benificing for the companies.
Unfortunately it mostly resulted in the presentation of the modification of the variables of a coefficient to obtain more precise reuslt when looking for an assesstment of companies return.
